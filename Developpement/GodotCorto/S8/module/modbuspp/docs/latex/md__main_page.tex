M\+O\+D\+B\+U\+S++ (M\+O\+D\+B\+U\+S\+PP) is an open source c++ class/ library making an encapsulation of \href{https://en.wikipedia.org/wiki/Modbus}{\tt Modbus} T\+CP Protocol published by Modicon (Now Schneider Electirc).

M\+O\+D\+B\+U\+S++ is based on Object-\/\+Oriented Programming. While it keeps the efficiency of C++ Code , it provides a higher level of abstraction than other C Modbus Library. Generally, it is easier for programmers to use in their development requiring M\+O\+D\+B\+US T\+CP Protocol. Compared to other known modbus library, such as libmodbus, M\+O\+D\+B\+U\+S++ is providing a more O\+OP friendly syntaxes.

The code has dependencies on libary on Linux for T\+C\+P/\+IP, if you want to use this in Windows, please check out winsock2 and rewrite portion of code of socket to be compatible with Windows operating system.

\section*{Usage}

To use the library, please follow the steps below. Please note current library is only compatible with Linux distributions because of the socket library dependencies.

\subsection*{Download}

Download the M\+O\+D\+B\+U\+S++, you can\+:
\begin{DoxyItemize}
\item Open your command window (cmd, shell, bash, etc.) \begin{quote}
git clone \href{https://github.com/fanzhe98/modbuspp.git}{\tt https\+://github.\+com/fanzhe98/modbuspp.\+git} \end{quote}

\item Directly Download From the \href{https://github.com/fanzhe98/modbuspp.git}{\tt Page}
\end{DoxyItemize}

\subsection*{Include In your code}

To include Modbus++ in your code, just simply type this in the start of your program\+: \begin{quote}
include \char`\"{}modbus.\+h\char`\"{} include \char`\"{}modbus\+\_\+exception.\+h\char`\"{} \end{quote}


\section*{Getting Started with a Example}

\subsection*{Getting the Example}

Please checkout example.\+cpp for a example usage of M\+O\+D\+B\+U\+S\+PP. Please note that the code should be compiled using c++11 compilers. \subsection*{Tutorials}

Include the header \begin{quote}
\#include \char`\"{}modbus.\+h\char`\"{} \end{quote}
This line includes the header \mbox{\hyperlink{modbus_8h_source}{modbus.\+h}} from the library, this will tell the compiler to look for functions and variables in the related file.

Create and connects a mobus server \begin{quote}
modbus mb = modbus(\char`\"{}127.\+0.\+0.\+1\char`\"{}, 502); mb.\+modbus\+\_\+set\+\_\+slave\+\_\+id(1); mb.\+modbus\+\_\+connect(); \end{quote}
These lines creates local modbus client with target IP address at 127.\+0.\+0.\+1 and port at 502, a server with id 1.

Read a coil (function 0x01) \begin{quote}
bool read\+\_\+coil; mb.\+modbus\+\_\+read\+\_\+coils(0, 1, \&read\+\_\+coil); \end{quote}
These lines read a bit from the coil at address 0.

Read input bits(discrete input) (function 0x02) \begin{quote}
bool read\+\_\+bits; mb.\+modbus\+\_\+read\+\_\+input\+\_\+bits(0, 1, \&read\+\_\+bits); \end{quote}
These lines read a bit from discrete inputs at address 0.

Read holding registers (function 0x03) \begin{quote}
uint16\+\_\+t read\+\_\+holding\+\_\+regs\mbox{[}1\mbox{]}; mb.\+modbus\+\_\+read\+\_\+holding\+\_\+registers(0, 1, read\+\_\+holding\+\_\+regs); \end{quote}
These lines read a word(16 bits) from holding registers starting at address 0.

Read input registers (function 0x04) \begin{quote}
uint16\+\_\+t read\+\_\+input\+\_\+regs\mbox{[}1\mbox{]}; mb.\+modbus\+\_\+read\+\_\+input\+\_\+registers(0, 1, read\+\_\+input\+\_\+regs); \end{quote}
These lines read a word(16 bits) from input registers starting at address 0.

Write single coil (function 0x05) \begin{quote}
mb.\+modbus\+\_\+write\+\_\+coil(0, true); \end{quote}
These lines write a bit to the coil at address 0.

Write single reg (function 0x06) \begin{quote}
mb.\+modbus\+\_\+write\+\_\+register(0, 123); \end{quote}
These lines write a word(16 bits) to the register at address 0.

Write multiple coils (function 0x0F) \begin{quote}
bool write\+\_\+cols\mbox{[}4\mbox{]} = \{true, true, true, true\}; mb.\+modbus\+\_\+write\+\_\+coils(0,4,write\+\_\+cols); \end{quote}
These lines write multiple bits to the coil starting from address 0, for a length of 4.

Write multiple regs (function 0x10) \begin{quote}
uint16\+\_\+t write\+\_\+regs\mbox{[}4\mbox{]} = \{123, 123, 123\}; mb.\+modbus\+\_\+write\+\_\+registers(0, 4, write\+\_\+regs); \end{quote}
These lines write multiple words to the register starting from address 0, for a length of 4. 