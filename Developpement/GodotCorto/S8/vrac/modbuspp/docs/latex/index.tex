M\+O\+D\+B\+U\+S++ (M\+O\+D\+B\+U\+S\+PP) is an open source c++ class/ library making an encapsulation of \href{https://en.wikipedia.org/wiki/Modbus}{\tt Modbus} T\+CP Protocol published by Modicon (Now Schneider Electirc).

M\+O\+D\+B\+U\+S++ is based on Object-\/\+Oriented Programming. While it keeps the efficiency of C++ Code , it provides a higher level of abstraction than other C Modbus Library. Generally, it is easier for programmers to use in their development requiring M\+O\+D\+B\+US T\+CP Protocol. Compared to other known modbus library, such as libmodbus, M\+O\+D\+B\+U\+S++ is providing a more O\+OP friendly syntaxes.

The code has dependencies on libary on Linux for T\+C\+P/\+IP, if you want to use this in Windows, please check out winsock2 and rewrite portion of code of socket to be compatible with Windows operating system. 



\section*{1 Usage}

To use the library, please follow the steps below. Please note current library is only compatible with Linux distributions because of the socket library dependencies.

\subsection*{1.\+1 Download}

Download the M\+O\+D\+B\+U\+S++, you can\+:
\begin{DoxyItemize}
\item Open your command window (cmd, shell, bash, etc.) $>$ git clone \href{https://github.com/fanzhe98/modbuspp.git}{\tt https\+://github.\+com/fanzhe98/modbuspp.\+git}
\item Directly Download From the \href{https://github.com/fanzhe98/modbuspp.git}{\tt Page}
\end{DoxyItemize}

\subsection*{1.\+2 Include In your code}

To include Modbus++ in your code, just simply type this in the start of your program\+: \begin{quote}
include \char`\"{}modbus.\+h\char`\"{} include \char`\"{}modbus\+\_\+exception.\+h\char`\"{} \end{quote}




\section*{2 Getting Started with a Example}

\subsection*{2.\+1 Getting the Example}

Checkout \href{https://github.com/fanzhe98/modbuspp/blob/master/example.cpp}{\tt example.\+cpp} for a example usage of M\+O\+D\+B\+U\+S\+PP. Please note that the code should be compiled using c++11 compilers. \subsection*{2.\+2 Tutorials}

Let\textquotesingle{}s break the code down, before you read this, BE S\+U\+RE to have a reference to \href{https://github.com/fanzhe98/modbuspp/blob/master/example.cpp}{\tt example.\+cpp}. The example shows basic usages of how to create a modbus client connecting to a modbus server and perform modbus operations in your program. \subsubsection*{Include the header}

To start with, be sure to include the \mbox{\hyperlink{modbus_8h_source}{modbus.\+h}} header in your program \begin{quote}
\#include \char`\"{}modbus.\+h\char`\"{} \end{quote}


This line includes the header \mbox{\hyperlink{modbus_8h_source}{modbus.\+h}} from the library, this will tell the compiler to look for functions and variables in the related file.

\subsubsection*{Create and connects a mobus server}

Before performing any modbus operations, a connection needed to be setup between the modbus client(your program) and a modbus server(could be a controller).

\begin{quote}
modbus mb = modbus(\char`\"{}127.\+0.\+0.\+1\char`\"{}, 502); mb.\+modbus\+\_\+set\+\_\+slave\+\_\+id(1); mb.\+modbus\+\_\+connect(); \end{quote}


These lines creates local modbus client with target IP address at 127.\+0.\+0.\+1 and port at 502, a server with id 1.

\subsubsection*{Performe Mobus Operations}

Modbus usually have 2 types of operations, reading and writing. Reading operations include reading a coil, reading input bits, reading holding registers, and reading input registers. Writing Operations include writing sigle coil, wirting single register, writing multiple coil, and writing multiple registers. Be sure to create a connection before performing any operations. Modbus is relatively low-\/level protocols. It is useful to have strong computer structure knowledge to understand it.


\begin{DoxyEnumerate}
\item The following line show how to read a coil (function 0x01). These lines read a bit from the coil at address 0. \begin{quote}
bool read\+\_\+coil; mb.\+modbus\+\_\+read\+\_\+coils(0, 1, \&read\+\_\+coil); \end{quote}

\item The following lines show how to read input bits(discrete input) (function 0x02). These lines read a bit from discrete inputs at address 0. \begin{quote}
bool read\+\_\+bits; mb.\+modbus\+\_\+read\+\_\+input\+\_\+bits(0, 1, \&read\+\_\+bits); \end{quote}

\item The follwing lines show how to read holding registers (function 0x03). These lines read a word(16 bits) from holding registers starting at address 0. \begin{quote}
uint16\+\_\+t read\+\_\+holding\+\_\+regs\mbox{[}1\mbox{]}; mb.\+modbus\+\_\+read\+\_\+holding\+\_\+registers(0, 1, read\+\_\+holding\+\_\+regs); \end{quote}

\item The following lines show how to Read input registers (function 0x04). These lines read a word(16 bits) from input registers starting at address 0. \begin{quote}
uint16\+\_\+t read\+\_\+input\+\_\+regs\mbox{[}1\mbox{]}; mb.\+modbus\+\_\+read\+\_\+input\+\_\+registers(0, 1, read\+\_\+input\+\_\+regs); \end{quote}

\item The following lines show how to write single coil (function 0x05). These lines write a bit to the coil at address 0. \begin{quote}
mb.\+modbus\+\_\+write\+\_\+coil(0, true); \end{quote}

\item The following lines show how to write single register (function 0x06). These lines write a word(16 bits) to the register at address 0. \begin{quote}
mb.\+modbus\+\_\+write\+\_\+register(0, 123); \end{quote}

\item The following lines show how to write multiple coils (function 0x0F). These lines write multiple bits to the coil starting from address 0, for a length of 4. \begin{quote}
bool write\+\_\+cols\mbox{[}4\mbox{]} = \{true, true, true, true\}; mb.\+modbus\+\_\+write\+\_\+coils(0,4,write\+\_\+cols); \end{quote}

\item The following lines show how to write multiple regs (function 0x10). These lines write multiple words to the register starting from address 0, for a length of 4. \begin{quote}
uint16\+\_\+t write\+\_\+regs\mbox{[}4\mbox{]} = \{123, 123, 123\}; mb.\+modbus\+\_\+write\+\_\+registers(0, 4, write\+\_\+regs);\end{quote}

\end{DoxyEnumerate}